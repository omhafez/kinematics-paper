\documentclass[12pt]{article}

%%%%%%%%%%%%%%%%%%%%%%%%%%%%%%%%%%%%%%%%%%%%%%%%%%%%%%%%%%%%%%%%%%%%%%%%
%                                                                      %
%               LATEX COMMANDS FOR DOCUMENT SETUP                      %
%                                                                      %
%%%%%%%%%%%%%%%%%%%%%%%%%%%%%%%%%%%%%%%%%%%%%%%%%%%%%%%%%%%%%%%%%%%%%%%%

%\usepackage{bookmark}
\usepackage[us,nodayofweek,12hr]{datetime}
\usepackage{graphicx}

\usepackage{tcolorbox}
\usepackage{fullpage}
\usepackage{graphicx}
\usepackage{tabularx}
\usepackage{multirow}
\usepackage{subfigure}
\usepackage{wrapfig}
\usepackage{textcomp}
\usepackage[square, comma, numbers, sort&compress]{natbib}
\usepackage[hang,small,bf]{caption}
\usepackage{parskip}
\usepackage[margin=1in,footskip=0.5in]{geometry}
\usepackage{amsmath}
\usepackage{mdwlist}
\usepackage{epstopdf}
\usepackage[section]{placeins}

\newcommand{\para}{\vspace{5mm} \noindent}
\newcommand{\parafig}{\vspace{4mm} \noindent}
\newcommand{\paraeq}{\vspace{1mm} \noindent}
\newcommand{\negparafig}{\vspace{-4mm} \noindent}
\newcommand{\negparaeq}{\vspace{-1mm} \noindent}

\addtolength{\parskip}{1\parskip}

\bibliographystyle{plain}
%\bibliographystyle{ieeetr}
%many other bibliography styles are available (IEEEtran, mla, etc.). Use one appropriate for your field.

\hyphenation{pre-par-ing} %add hyphenation rules for words TeX doesn't know

%\usepackage[square,comma,numbers,sort&compress]{natbib}
%\usepackage{hypernat}
% Other useful packages to try
\usepackage{amsmath}
\usepackage{amssymb}
\usepackage{accents}

\usepackage{setspace}
\usepackage{algorithm}
\usepackage{algorithmic}
%
% Different fonts to try (uncomment only fontenc and one font at a time)
% (you may need to install these first)
%\usepackage[T1]{fontenc} %enable fontenc package if using one of the fonts below
%\usepackage[adobe-utopia]{mathdesign}
%\usepackage{tgschola}
%\usepackage{tgbonum}
%\usepackage{tgpagella}
%\usepackage{tgtermes}
%\usepackage{fourier}
%\usepackage{fouriernc}
%\usepackage{kmath,kerkis}
%\usepackage{kpfonts}
%\usepackage[urw-garamond]{mathdesign}
%\usepackage[bitstream-charter]{mathdesign}
%\usepackage[sc]{mathpazo}
%\usepackage{mathptmx}
%\usepackage[varg]{txfonts}


%%%%%%%%%%%%%%%%%%%%%%%%%%%%%%%%%%%%%%%%%%%%%%%%%%%%%%%%%%%%%%%%%%%%%%%%
%                                                                      %
%        DOCUMENT SETUP AND INFORMATION FOR PRELIMINARY PAGES          %
%                                                                      %
%%%%%%%%%%%%%%%%%%%%%%%%%%%%%%%%%%%%%%%%%%%%%%%%%%%%%%%%%%%%%%%%%%%%%%%%

\begin{document}

%%%%%%%%%%%%%%%%%%%%%%%%%%%%%%%%%%%%%%%%%%%%%%%%%%%%%%%%%%%%%%%%%%%%%%%%

\vspace{7cm}
\begin{center}
   {\bf\Large An Incremental Kinematics Implementation for Finite Element Applications in Finite Deformation Continuum Mechanics}
\end{center}

\vspace{1.5cm}
\begin{center} 
{\bf\large Omar Hafez, Brian Giffin, Sam Mish, Mark Rashid}\\

%\vspace{5mm}
{Department of Civil \& Environmental Engineering}\\
{University of California, Davis}\end{center}

\abstract{The use of hypoelastic material models for applications in computational solid mechanics poses several challenges when finite deformations are taken into consideration. Namely, an appropriate objective (co-rotational) stress rate must be utilized, and a method of approximately integrating the chosen rate equation over a finite time interval must be devised. These concerns have led to the development of so-called ``incrementally objective'' algorithms, which provide a means of updating the state of stress in a material over successive increments of deformation, such that the approximation remains consistent with the particular choice of objective stress rate. Nonetheless, certain approaches may incur accumulating errors over time which can become arbitrarily large, particularly for problems involving cyclic deformations. This paper presents a ``strongly objective'' stress update algorithm based upon the Jaumann rate of stress, which relies upon an efficient polar decomposition of the deformation gradient to retain accuracy in evaluating finite element residual and consistent tangent contributions. The accuracy of the method is compared against other approaches commonly found in commercial finite element software, and its nonlinear convergence performance is examined in relation to other methods which lack algorithmic consistency.}

%%%%%%%%%%%%%%%%%%%%%%%%%%%%%%%%%%%%%%%%%%%%%%%%%%%%%%%%%%%%%%%%%%%%%%%%

% Each chapter can be in its own file for easier editing and brought in with the \include command.
% Then use the \includeonly command to speed compilation when working on a particular chapter.
%%% \includeonly{chap1}

%\newcommand{\bibfont}{\singlespacing}
% need this command to keep single spacing in the bibliography when using natbib

\section{Introduction}

Submit to CMAME \\


{\bf OUTLINE} \\

\begin{itemize}
\item {\bf Introduction (MARK)}
\begin{itemize}
\item finite deformation continuum mechanics, interest in application in finite elements
\item Lagrangian mechanics, with large but reasonable deformations, not highly compressible, not 2000\% strain
\item then need finite deformation capable incremental kinematics
\item describe specific format for incremental kinematics, i.e., we want D
\item take in incremental displacements, want D and Rhat
\item need to justify this
\item if you have a corotational rate, it *should* be a Jaumann rate
\item need for accuracy in incremental kinematics
\item coarse approximations are not sufficient
\item discuss Hughes-Winget and Rashid 1993 / literature review of other approaches and why they aren't good enough
\item Hughes-Winget: solving the exact differential, but solve approximately
\item Ours: approximate the differential, and solve exactly
\item mention we've set up some verification problems for your code / demonstration of the algorithm
\item interested in implementing finite element codes
\item consistent tangent to get good convergence
\item literature review of incremental kinematics
\end{itemize}

\item {\bf Incremental Kinematics / Integration of Co-Rotational Rates / Kinematic Splitting}
\begin{itemize}
\item problem statement, what are we trying to do
\item decompose the motion, need to separate stretch and rotate
\item describe with equation Rashid 93 and Hughes Winget
\item explain in detail what we do
\item present the algorithm
\end{itemize}

\item {\bf Finite Element Implementation}
\begin{itemize}
\item work based on my previous document
\item take a look at Brian's stuff
\item explain we're interested in finite element applications
\item involves the residual
\item where does the material update fit into all of this
\item motivate why need derivatives, newton's method
\item high level equations for that
\item appendix: more detailed equations if necessary
\item KEEP IN MIND: may not be necessary to zoom out too far to residuals. Derivatives with respect to Fhat better. Would allow us to not mention Piola
\item full accounting of tangent modulus
\item traction and pressure BC terms as well
\item grab new BC terms from Sam
\end{itemize}

\item {\bf demonstration of accuracy, comparison to code A}
\begin{itemize}
\item describe all the problems
\item results
\item interpretations
\end{itemize}

\item {\bf convergence}
\begin{itemize}
\item show importance of including all of the terms required for tangent stiffness as it pertains to convergence
\item turn off selected terms in tangent stiffness, for example dRhat/duhat
\item results
\item interpretations
\end{itemize}

\item {\bf conclusions}

\end{itemize}


make comment about fatigue - lots of cycles

\subsection{Explanation of References}
Hughes and Winget have proposed one of the earliest (weakly) objective incremental algorithms in \cite{hughes1980} which relies upon the Jaumann rate of stress. Therein, they introduced the notion of ``incremental objectivity.'' Subsequently, Rubinstein and Atluri provided a more rigorous mathematical investigation into the conditions of objectivity for incremental algorithms in \cite{rubinstein1983}. Flanagan and Taylor later proposed an algorithm in \cite{flanagan1987} based upon the Green-Naghdi co-rotational rate, which considers an evolution of the material state in its ``unrotated'' configuration.

In contrast with previous works where both the stretching and rotation increment are computed approximately, Roy et. al. have suggested in \cite{roy1992} that the polar decomposition be computed via the Cayley-Hamilton theorem for use in incrementally objective algorithms to improve accuracy.

Rashid introduced the notion of ``strong'' objectivity in the context of incrementally objective algorithms in \cite{rashid1993}. A strongly objective algorithm was then proposed, though it utilized approximate expressions for the stretching and rotation tensors appearing therein. Nonetheless, significant improvements in accuracy were demonstrated by Rashid and Thorne in \cite{rashid1996}, particularly for cyclic shearing deformations.

Guo has developed relatively simple expressions for the rates of stretch and rotation tensors in \cite{guo1984}, which may be easily generalized in order to express the derivatives of these quantities with respect to other tensors of interest. Hoger and Carlson followed up on these developments in \cite{hoger1984}, where they developed an alternative set of expressions for these same quantities. Carlson and Hoger would later develop yet more general expressions for the derivatives of tensor-valued functions with respect to other tensors in \cite{carlson1986}, though these are quite convoluted.

Hoger developed an expression for the time rate of logarithmic strain in \cite{hoger1986}, and Ba\v{z}ant has proposed in \cite{bazant1998} an approximate expression for the Hencky (logarithmic) strain and its rate. Jog ultimately arrived at an explicit representation for the logarithm of a tensor and its derivatives in \cite{jog2008} which relies upon the eigendecomposition.

In \cite{danielson2014}, Danielson has suggested an approach based on successive Newton-Raphson iteration to obtain the polar decomposition of the deformation gradient, for prospective use in various kinematic update algorithms. Alternatively, Scherzinger and Dohrmann have proposed a relatively fast and highly accurate method for determining the eigendecomposition of 3 $\times$ 3 symmetric matricies in \cite{scherzinger2008}, which may be used to readily compute many quantities necessary to the incremental stress update procedure.

Fish and Shek have investigated the importance of adopting a consistent linearization of the incrementally objective algorithm of Hughes and Winget in \cite{fish1999}, demonstrating very poor solution convergence when an approximate tangent is utilized instead.

Kamojjala et. al. have proposed a series of solid mechanics verification problems in \cite{kamojjala2015}, which includes a test for frame indifference (i.e. weak objectivity), though a test for verification of strong objectivity is conspicuously absent.

In \cite{Khoei2003}, Khoei et. al. have developed a specialized incrementally objective algorithm for endochronic constitutive models which uses time sub-increments to integrate the stress rate equations, with the strain increment in each sub-interval obtained via the midpoint rule proposed by Hughes and Winget.

Rubin and Papes discuss the formulation of incrementally objective algorithms for use with elastic-viscoplastic constitutive models in \cite{rubin2011}, noting a clear advantage of strongly objective algorithms over weakly objective algorithms in that context.

Xiao, Bruhn, and Meyers have argued in \cite{xiao1997} and \cite{xiao1998} for the use of the so-called ``logarithmic'' stress rate over other co-rotational rates, citing that such a rate yields work conjugate stress and strain measures. Zhou and Tamma have developed an incrementally objective algorithm based upon the the logarithmic rate in \cite{zhou2003}, though their formulation utilizes a mid-point approximation scheme to compute the stretch increment. Moreover, little to no discussion is given regarding the computation of a consistent tangent for the proposed algorithm.


\section{Incremental Kinematics}
%


\newcommand{\ds}{\displaystyle}
\newcommand{\mparen}[1]{\displaystyle \big( #1 \big)}
\newcommand{\mbrack}[1]{\displaystyle \big[ #1 \big]}
\newcommand{\bparen}[1]{\displaystyle \bigg( #1 \bigg)}
\newcommand{\bbrack}[1]{\displaystyle \bigg[ #1 \bigg]}

The Jaumann rate of stress for a linear hypoelastic material is characterized by:

$$ \overset{\Delta}{\sigma} \equiv 
\dot{\sigma} + \sigma \cdot W - W \cdot \sigma = 
C : D $$

where $ D = \frac{1}{2} (L + L^\top), \; W = \frac{1}{2} (L - L^\top)$, and $L = \dot{F}
F^{-1} $. We would like to know how the stress state changes after applying
a prescribed deformation. Our approach is as follows: parameterize the deformation
into separate stages of pure stretch and pure rotation. The advantage of this
approach is that it allows the terms in the stress rate above to be integrated
separately

\begin{align*}
\dot{\sigma} = 
\begin{cases}
C : D & \qquad \text{W vanishes for pure stretch} \\
W \cdot \sigma - \sigma \cdot W & \qquad \text{D vanishes for pure rotation}
\end{cases}
\end{align*}

Simple analytic solutions exist for each of these ordinary differential equations, 
provided that $D, W$ are constant ($C$ is also assumed to be constant):

\begin{align*}
\dot{\sigma}(t) = C : D, \; \sigma(0) = \sigma_i 
\qquad 
& \Longrightarrow 
\qquad 
\sigma_{i+\frac{1}{2}}  = \sigma_i + C : D \Delta t \\ \\
\dot{\sigma}(t) = W \cdot \sigma - \sigma \cdot W, \; \sigma(0) = \sigma_{i+\frac{1}{2}} 
\qquad 
& \Longrightarrow 
\qquad 
\sigma_{i+1} = \exp(W \Delta t) \; \sigma_{i+\frac{1}{2}} \; \exp(W \Delta t)^{\top}
\end{align*}

Combining these two results gives us our stress update procedure:

$$ \sigma_{i+1} = \exp(W \Delta t) \; \bparen{\sigma_i + C : D \Delta t} \; \exp(W \Delta t)^{\top} $$

All that remains is to find reasonable approximations for $D, W$. To that end, we
consider the polar decomposition of the incremental deformation gradient, 
$F = R \; U$. After taking a time derivative of $F$, and substituting into the
definition of $L$, we get:

$$ L \equiv D + W = \dot{R} \; R^\top + R \; \dot{U} \; U^{-1} \; R^\top $$

When considering the separate stages of pure stretch and pure rotation, this
equation reduces to:

\begin{align*}
\begin{cases}
D = \dot{U} \; U^{-1} & \qquad \text{pure stretch} \\
W = \dot{R} \; R^\top & \qquad \text{pure rotation}
\end{cases}
\end{align*}

These are constant-coefficient, ordinary differential equations
than can be used to determine $D$ and $W$. The
general solution for equations of this form is

$$ 
Y = \dot{X} X^{-1}, \; X(0) = \mathbf{1} 
\qquad 
\Longrightarrow 
\qquad 
X(\Delta t) = \exp(Y \Delta t)
$$

Which, when applied to the pure stretch and pure rotation versions of the problem
gives us our definitions of $D, W$:

$$ D = \frac{1}{\Delta t} \log(U), \qquad W = \frac{1}{\Delta t} \log(R) $$

Substituting these back into our stress update procedure gives 

$$\boxed{\sigma_{i+1} = R \; \bparen{\sigma_i + C : \log(U)} \; R^\top}$$

As a practical matter, an implementation of this procedure is included below:


\begin{tcolorbox}
Given the current deformation gradient $\mathbf{F}(t)$, current Cauchy stress $\sigma(t)$, and
a displacement increment $\hat{u}$,
\begin{enumerate}
\item compute the end-step deformation gradient
$$ \mathbf{F}(t + \Delta t) = \mathbf{F}(t) + \nabla \hat{u} $$
\item find the incremental deformation gradient
$$ \Delta \mathbf{F} = \mathbf{F}(t + \Delta t) \, \mathbf{F}(t)^{-1} $$
\item get the strain increment
$$ \Delta \mathbf{E} = \frac{1}{2}(\Delta \mathbf{F}^{\,\intercal} \Delta \mathbf{F} - \mathbf{1}) $$
\item Determine $\mathbf{Q}, \mathbf{\Lambda}$ from the eigendecomposition\footnote{
a closed form expression for the eigendecomposition of 3x3 symmetric matrices can be found 
in this paper by ...
} of $\Delta \mathbf{E}$
$$ \frac{1}{2} \mathbf{Q} \, (\mathbf{\Lambda}^2 - \mathbf{1}) \, \mathbf{Q}^\intercal = \Delta \mathbf{E} $$
\item form the polar decomposition of $\Delta \mathbf{F} = \Delta \mathbf{R} \Delta \mathbf{U}$
$$ 
\begin{cases} 
  \Delta \mathbf{U} = \mathbf{Q} \, \mathbf{\Lambda} \, \mathbf{Q}^\intercal  \\
  \Delta \mathbf{R} = \Delta \mathbf{F} \; \Delta \mathbf{U}^{-1}
\end{cases}
$$
\item compute the updated, unrotated stress
$$ \bar \sigma = 
\hat{\sigma}(\sigma(t), \log(\Delta \mathbf{U})) = 
\hat{\sigma}(\sigma(t), \mathbf{Q} \, \log(\mathbf{\Lambda}) \, \mathbf{Q}^\intercal)
$$
\item rotate the stress
$$ \sigma(t + \Delta t) = \Delta \mathbf{R} \; \bar{\sigma} \; \Delta \mathbf{R}^\intercal $$
\end{enumerate}
\end{tcolorbox}

\pagebreak

\subsection{Kinematic Splitting Algorithm}

The specification of an appropriate kinematic splitting algorithm depends on two primary considerations: the objective stress rate with which the intended algorithm is to maintain consistency, and the desired level of accuracy that the algorithm should achieve.


%%%%%%%%%%%%%%%%%%%%%%%%%%%%%%%%%%%%%%%%%%%%%%%
%%%%%%%%%%%%%%%%%%%%%%%%%%%%%%%%%%%%%%%%%%%%%%%
\subsection{Description}
\label{Description}

%%%%%%%%%%%%%%%%%%%%%%%%%%%%%%%%%%%%%%%%%%%%%%%
%%%%%%%%%%%%%%%%%%%%%%%%%%%%%%%%%%%%%%%%%%%%%%%
\subsection{Implementation Approaches}
\label{Implementation Approaches}
%%%%%%%%%%%%%%%%%%%%%%%%%%%%%%%%%%%%%%%%%%%%%%%
%%%%%%%%%%%%%%%%%%%%%%%%%%%%%%%%%%%%%%%%%%%%%%%
\subsubsection{Hughes-Winget}
\label{Hughes-Winget}

\subsubsection{Rashid}
\label{Rashid}

hi
\include{chap3-implementation}
\include{chap4-accuracy}
\include{chap5-convergence}
\include{chap6-conclusions}
\include{chap7-appendix}

%%%%%%%%%%%%%%%%%%%%%%%%%%%%%%%%%%%%%%%%%%%%%%%%%%%%%%%%%%%%%%%%%%%%%%%%

\bibliography{references}

%%%%%%%%%%%%%%%%%%%%%%%%%%%%%%%%%%%%%%%%%%%%%%%%%%%%%%%%%%%%%%%%%%%%%%%%

\end{document}