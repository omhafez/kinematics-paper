\section{Incremental Kinematics}
%


\newcommand{\ds}{\displaystyle}
\newcommand{\mparen}[1]{\displaystyle \big( #1 \big)}
\newcommand{\mbrack}[1]{\displaystyle \big[ #1 \big]}
\newcommand{\bparen}[1]{\displaystyle \bigg( #1 \bigg)}
\newcommand{\bbrack}[1]{\displaystyle \bigg[ #1 \bigg]}

The Jaumann rate of stress for a linear hypoelastic material is characterized by:

$$ \overset{\Delta}{\sigma} \equiv 
\dot{\sigma} + \sigma \cdot W - W \cdot \sigma = 
C : D $$

where $ D = \frac{1}{2} (L + L^\top), \; W = \frac{1}{2} (L - L^\top)$, and $L = \dot{F}
F^{-1} $. We would like to know how the stress state changes after applying
a prescribed deformation. Our approach is as follows: parameterize the deformation
into separate stages of pure stretch and pure rotation. The advantage of this
approach is that it allows the terms in the stress rate above to be integrated
separately

\begin{align*}
\dot{\sigma} = 
\begin{cases}
C : D & \qquad \text{W vanishes for pure stretch} \\
W \cdot \sigma - \sigma \cdot W & \qquad \text{D vanishes for pure rotation}
\end{cases}
\end{align*}

Simple analytic solutions exist for each of these ordinary differential equations, 
provided that $D, W$ are constant ($C$ is also assumed to be constant):

\begin{align*}
\dot{\sigma}(t) = C : D, \; \sigma(0) = \sigma_i 
\qquad 
& \Longrightarrow 
\qquad 
\sigma_{i+\frac{1}{2}}  = \sigma_i + C : D \Delta t \\ \\
\dot{\sigma}(t) = W \cdot \sigma - \sigma \cdot W, \; \sigma(0) = \sigma_{i+\frac{1}{2}} 
\qquad 
& \Longrightarrow 
\qquad 
\sigma_{i+1} = \exp(W \Delta t) \; \sigma_{i+\frac{1}{2}} \; \exp(W \Delta t)^{\top}
\end{align*}

Combining these two results gives us our stress update procedure:

$$ \sigma_{i+1} = \exp(W \Delta t) \; \bparen{\sigma_i + C : D \Delta t} \; \exp(W \Delta t)^{\top} $$

All that remains is to find reasonable approximations for $D, W$. To that end, we
consider the polar decomposition of the incremental deformation gradient, 
$F = R \; U$. After taking a time derivative of $F$, and substituting into the
definition of $L$, we get:

$$ L \equiv D + W = \dot{R} \; R^\top + R \; \dot{U} \; U^{-1} \; R^\top $$

When considering the separate stages of pure stretch and pure rotation, this
equation reduces to:

\begin{align*}
\begin{cases}
D = \dot{U} \; U^{-1} & \qquad \text{pure stretch} \\
W = \dot{R} \; R^\top & \qquad \text{pure rotation}
\end{cases}
\end{align*}

These are constant-coefficient, ordinary differential equations
than can be used to determine $D$ and $W$. The
general solution for equations of this form is

$$ 
Y = \dot{X} X^{-1}, \; X(0) = \mathbf{1} 
\qquad 
\Longrightarrow 
\qquad 
X(\Delta t) = \exp(Y \Delta t)
$$

Which, when applied to the pure stretch and pure rotation versions of the problem
gives us our definitions of $D, W$:

$$ D = \frac{1}{\Delta t} \log(U), \qquad W = \frac{1}{\Delta t} \log(R) $$

Substituting these back into our stress update procedure gives 

$$\boxed{\sigma_{i+1} = R \; \bparen{\sigma_i + C : \log(U)} \; R^\top}$$

As a practical matter, an implementation of this procedure is included below:


\begin{tcolorbox}
Given the current deformation gradient $\mathbf{F}(t)$, current Cauchy stress $\sigma(t)$, and
a displacement increment $\hat{u}$,
\begin{enumerate}
\item compute the end-step deformation gradient
$$ \mathbf{F}(t + \Delta t) = \mathbf{F}(t) + \nabla \hat{u} $$
\item find the incremental deformation gradient
$$ \Delta \mathbf{F} = \mathbf{F}(t + \Delta t) \, \mathbf{F}(t)^{-1} $$
\item get the strain increment
$$ \Delta \mathbf{E} = \frac{1}{2}(\Delta \mathbf{F}^{\,\intercal} \Delta \mathbf{F} - \mathbf{1}) $$
\item Determine $\mathbf{Q}, \mathbf{\Lambda}$ from the eigendecomposition\footnote{
a closed form expression for the eigendecomposition of 3x3 symmetric matrices can be found 
in this paper by ...
} of $\Delta \mathbf{E}$
$$ \frac{1}{2} \mathbf{Q} \, (\mathbf{\Lambda}^2 - \mathbf{1}) \, \mathbf{Q}^\intercal = \Delta \mathbf{E} $$
\item form the polar decomposition of $\Delta \mathbf{F} = \Delta \mathbf{R} \Delta \mathbf{U}$
$$ 
\begin{cases} 
  \Delta \mathbf{U} = \mathbf{Q} \, \mathbf{\Lambda} \, \mathbf{Q}^\intercal  \\
  \Delta \mathbf{R} = \Delta \mathbf{F} \; \Delta \mathbf{U}^{-1}
\end{cases}
$$
\item compute the updated, unrotated stress
$$ \bar \sigma = 
\hat{\sigma}(\sigma(t), \log(\Delta \mathbf{U})) = 
\hat{\sigma}(\sigma(t), \mathbf{Q} \, \log(\mathbf{\Lambda}) \, \mathbf{Q}^\intercal)
$$
\item rotate the stress
$$ \sigma(t + \Delta t) = \Delta \mathbf{R} \; \bar{\sigma} \; \Delta \mathbf{R}^\intercal $$
\end{enumerate}
\end{tcolorbox}

\pagebreak

\subsection{Kinematic Splitting Algorithm}

The specification of an appropriate kinematic splitting algorithm depends on two primary considerations: the objective stress rate with which the intended algorithm is to maintain consistency, and the desired level of accuracy that the algorithm should achieve.


%%%%%%%%%%%%%%%%%%%%%%%%%%%%%%%%%%%%%%%%%%%%%%%
%%%%%%%%%%%%%%%%%%%%%%%%%%%%%%%%%%%%%%%%%%%%%%%
\subsection{Description}
\label{Description}

%%%%%%%%%%%%%%%%%%%%%%%%%%%%%%%%%%%%%%%%%%%%%%%
%%%%%%%%%%%%%%%%%%%%%%%%%%%%%%%%%%%%%%%%%%%%%%%
\subsection{Implementation Approaches}
\label{Implementation Approaches}
%%%%%%%%%%%%%%%%%%%%%%%%%%%%%%%%%%%%%%%%%%%%%%%
%%%%%%%%%%%%%%%%%%%%%%%%%%%%%%%%%%%%%%%%%%%%%%%
\subsubsection{Hughes-Winget}
\label{Hughes-Winget}

\subsubsection{Rashid}
\label{Rashid}

hi