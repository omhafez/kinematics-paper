\section{Introduction}

Submit to CMAME \\

1. justification for our specific format for incremental kinematics, i.e., we want D \\
  - take in incremental displacements, want D and Rhat \\
  - need to justify this \\
  - if you have a corotational rate, it *should* be a Jaumann rate \\
2. computation of D and Rhat \\
3. full accounting of tangent modulus \\
4. traction and pressure BC terms as well \\
5. systematic exposition of accuracy, comparison to code A \\
6. show importance of including all of the terms required for tangent stiffness as it pertains to convergence \\
  - turn off selected terms in tangent stiffness, for example dRhat/duhat \\

make comment about fatigue - lots of cycles

\subsection{Explanation of References}
Hughes and Winget have proposed one of the earliest (weakly) objective incremental algorithms in \cite{hughes1980} which relies upon the Jaumann rate of stress. Therein, they introduced the notion of ``incremental objectivity.'' Subsequently, Rubinstein and Atluri provided a more rigorous mathematical investigation into the conditions of objectivity for incremental algorithms in \cite{rubinstein1983}. Flanagan and Taylor later proposed an algorithm in \cite{flanagan1987} based upon the Green-Naghdi co-rotational rate, which considers an evolution of the material state in its ``unrotated'' configuration.

In contrast with previous works where both the stretching and rotation increment are computed approximately, Roy et. al. have suggested in \cite{roy1992} that the polar decomposition be computed via the Cayley-Hamilton theorem for use in incrementally objective algorithms to improve accuracy.

Rashid introduced the notion of ``strong'' objectivity in the context of incrementally objective algorithms in \cite{rashid1993}. A strongly objective algorithm was then proposed, though it utilized approximate expressions for the stretching and rotation tensors appearing therein. Nonetheless, significant improvements in accuracy were demonstrated by Rashid and Thorne in \cite{rashid1996}, particularly for cyclic shearing deformations.

Guo has developed relatively simple expressions for the rates of stretch and rotation tensors in \cite{guo1984}, which may be easily generalized in order to express the derivatives of these quantities with respect to other tensors of interest. Hoger and Carlson followed up on these developments in \cite{hoger1984}, where they developed an alternative set of expressions for these same quantities. Carlson and Hoger would later develop yet more general expressions for the derivatives of tensor-valued functions with respect to other tensors in \cite{carlson1986}, though these are quite convoluted.

Hoger developed an expression for the time rate of logarithmic strain in \cite{hoger1986}, and Ba\v{z}ant has proposed in \cite{bazant1998} an approximate expression for the Hencky (logarithmic) strain and its rate. Jog ultimately arrived at an explicit representation for the logarithm of a tensor and its derivatives in \cite{jog2008} which relies upon the eigendecomposition.

In \cite{danielson2014}, Danielson has suggested an approach based on successive Newton-Raphson iteration to obtain the polar decomposition of the deformation gradient, for prospective use in various kinematic update algorithms. Alternatively, Scherzinger and Dohrmann have proposed a relatively fast and highly accurate method for determining the eigendecomposition of 3 $\times$ 3 symmetric matricies in \cite{scherzinger2008}, which may be used to readily compute many quantities necessary to the incremental stress update procedure.

Fish and Shek have investigated the importance of adopting a consistent linearization of the incrementally objective algorithm of Hughes and Winget in \cite{fish1999}, demonstrating very poor solution convergence when an approximate tangent is utilized instead.

Kamojjala et. al. have proposed a series of solid mechanics verification problems in \cite{kamojjala2015}, which includes a test for frame indifference (i.e. weak objectivity), though a test for verification of strong objectivity is conspicuously absent.

In \cite{Khoei2003}, Khoei et. al. have developed a specialized incrementally objective algorithm for endochronic constitutive models which uses time sub-increments to integrate the stress rate equations, with the strain increment in each sub-interval obtained via the midpoint rule proposed by Hughes and Winget.

Rubin and Papes discuss the formulation of incrementally objective algorithms for use with elastic-viscoplastic constitutive models in \cite{rubin2011}, noting a clear advantage of strongly objective algorithms over weakly objective algorithms in that context.

Xiao, Bruhn, and Meyers have argued in \cite{xiao1997} and \cite{xiao1998} for the use of the so-called ``logarithmic'' stress rate over other co-rotational rates, citing that such a rate yields work conjugate stress and strain measures. Zhou and Tamma have developed an incrementally objective algorithm based upon the the logarithmic rate in \cite{zhou2003}, though their formulation utilizes a mid-point approximation scheme to compute the stretch increment. Moreover, little to no discussion is given regarding the computation of a consistent tangent for the proposed algorithm.

